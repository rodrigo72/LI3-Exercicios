\documentclass[12pt]{article}
\usepackage[utf8]{inputenc}
\usepackage{listings}
\usepackage{color}

\definecolor{dkgreen}{rgb}{0,0.6,0}
\definecolor{gray}{rgb}{0.5,0.5,0.5}
\definecolor{mauve}{rgb}{0.58,0,0.82}

\lstset{frame=tb,
  language=C,
  aboveskip=3mm,
  belowskip=3mm,
  showstringspaces=false,
  columns=flexible,
  basicstyle={\small\ttfamily},
  numbers=none,
  numberstyle=\tiny\color{gray},
  keywordstyle=\color{blue},
  commentstyle=\color{dkgreen},
  stringstyle=\color{mauve},
  breaklines=true,
  breakatwhitespace=true,
  tabsize=3
}

\title{Laboratórios de informática III}
\author{Rodrigo Monteiro}
\date{Outubro 2022}


\begin{document}

\maketitle
\pagebreak

\begin{center}
    {\huge Guião 2 - Exercício} \\[8pt]
\end{center}

\section{Files}

\begin{verbatim}
    commands.c
    commands.h
    deque.c
    deque.h
    input.txt
    LaTex
    main.c
    Makefile
\end{verbatim}

\section{Makefile}

\begin{lstlisting}
    CC = gcc
    CFLAGS = -Wall -O2
    LIBS = -lm
    OBJS = main.o deque.o commands.o
    TARGET = main
    ARGS = input.txt

    $(TARGET): $(OBJS)
    	$(CC) $(CFLAGS) -o $@ $^ $(LIBS)

    run: $(TARGET)
    	./main $(ARGS)

    clean:
    	rm -f $(TARGET) $(OBJS)
\end{lstlisting}

\pagebreak

\section{Structs}

\begin{lstlisting}
    // commands.h
    typedef struct cmd {
        char *command;
        int *args; // NULL se nao houver
        int nargs; // Numero de argumentos
    } Cmd;

    // deque.h
    typedef struct node {
        void *data;
        struct node *prev;
        struct node *next;
    } Node;

    typedef struct deque {
        int size;
        Node *first;
        Node *last;
        bool reversed;
    } Deque;
\end{lstlisting}

\end{document}